\documentclass{article}

\usepackage[utf8]{inputenc}
\title{Matematica Computazionale Progetto}
\date{}

\begin{document}
	
	\maketitle

	\section*{Design}
	
	HomePage che permette di andare in ogni angolo dell'applicazione.
		
	Teoria divisa in 4 parti (in ordine):
	
	\begin{enumerate}
		\item Integrali Definiti
		\item Integrali Indefiniti
		\item Teorema Fondamentale
		\item Metodi per la risoluzione (Divisi in Definiti e Indefiniti)
	\end{enumerate}
	Dalla HomePage si può accedere al primo livello di teoria che sono (1) e (2), poi al secondo livello che riguarda (3) e infine al terzo livello: (4).\\
	
	Idealmente lo studente non può accedere arbitrariamente a qualunque parte finché non ha studiato in ordine ogni livello.\\
	\\
	Sempre dalla HomePage si può accedere alla sezione degli Esercizi, ed eventualmente al quiz/gioco. (se lo faremo)\\
	\\
	Dopo la Teoria, Esercizi. Divisi in due tipologie 
	
	\begin{enumerate}
		\item Su integrali definiti
		\begin{enumerate}
			\item Per sostituzione
			\item Per parti
			\item ...
		\end{enumerate}
		\item Su integrali indefiniti
		\begin{enumerate}
			\item Per sostituzione
			\item Per parti
			\item ...
		\end{enumerate}
	\end{enumerate}
	Potremmo fare anche noi una specie di quiz/gioco a punteggio con vari tipi di esercizi:
	\begin{itemize}
		\item Mostriamo grafico e scegliere tra risposte multiple
		\item Diamo espressione da svolgere
		\item Volume dei solidi di rotazione
	\end{itemize}

	
	\subsection*{Info}
	\begin{itemize}
		
		\item In ogni lezione esempio di svolgimento di esercizi con suggerimenti su cosa si sta facendo. (magari interattivo, cioè può passare da un passaggio della soluzione all'altro invece di mostrare tutto staticamente)
	
		\item Nella sezione degli esercizi si può suggerire il metodo da utilizzare se lo studente chiede un aiuto.
		
		\item Nella sezione degli esercizi lo studente può vedere ogni passaggio opzionalmente. Ad esempio se si blocca e ha bisogno di un suggerimento, premendo un bottone può vedere il passaggio successivo e da li continuare da solo.
		
		\item Magari potremmo fare una tipologia di esercizi dove non si può vedere la soluzione nè chiedere il suggerimento sul metodo di risoluzione.
				
	\end{itemize}


\subsection*{Teoria Integrali Indefiniti}
	\begin{itemize}
	\item primitiva
	\item  lineare
	\item D($\int f$) = $f$
	\item esempi su funzione elementare
	\end{itemize}

\subsection*{Esercizi Integrali Indefiniti}

	\begin{itemize}
		\item Integrale elementare
		\item Scomposizione
		\item Sostituzione
		\item Per parti
		\item f. razionale
	\end{itemize}

\subsection*{Esercizi Integrali Definiti}

\begin{itemize}
	\item Integrale elementare
	\item Scomposizione
	\item Additività
	\item Sostituzione
	\item Parti
\end{itemize}

\end{document}